\section{45. Select your language}
The messages in Vim can be given in several languages.
This chapter explains how to change which one is used.
Also, the different ways to work with files in various languages is explained.
\localtableofcontentswithrelativedepth{+1}
\subsection{Language for Messages}
When you start Vim, it checks the environment to find out what language you are using.
Mostly this should work fine, and you get the messages in your language (if they are available).
To see what the current language is, use this command:

\begin{Verbatim}[samepage=true]
 :language
\end{Verbatim}

If it replies with "\texttt{C}", this means the default is being used, which is English.

\emph{Note}:
Using different languages only works when Vim was compiled to handle it.
To find out if it works, use the "\texttt{:version}" command and check the output for "\texttt{+gettext}" and "\texttt{+multi\_lang}".
If they are there, you are OK.
If you see "\texttt{-gettext}" or "\texttt{-multi\_lang}" you will have to find another Vim.

What if you would like your messages in a different language?  There are several ways.
Which one you should use depends on the capabilities of your system.

The first way is to set the environment to the desired language before starting Vim.
Example for Unix:

\begin{Verbatim}[samepage=true]
 env LANG=de_DE.ISO_8859-1  vim
\end{Verbatim}

This only works if the language is available on your system.
The advantage is that all the GUI messages and things in libraries will use the right language as well.
A disadvantage is that you must do this before starting Vim.
If you want to change language while Vim is running, you can use the second method:

\begin{Verbatim}[samepage=true]
 :language fr_FR.ISO_8859-1
\end{Verbatim}

This way you can try out several names for your language.
You will get an error message when it's not supported on your system.
You don't get an error when translated messages are not available.
Vim will silently fall back to using English.

To find out which languages are supported on your system, find the directory where they are listed.
On my system it is "\texttt{/usr/share/locale}".
On some systems it's in "\texttt{/usr/lib/locale}".
The manual page for "\texttt{setlocale}" should give you a hint where it is found on your system.

Be careful to type the name exactly as it should be.
Upper and lowercase matter, and the '\texttt{-}' and '\texttt{\_}' characters are easily confused.

You can also set the language separately for messages, edited text and the time format.
See |\texttt{:h :language}|.

\subsubsection{Do-it-yourself message translation}
If translated messages are not available for your language, you could write them yourself.
To do this, get the source code for Vim and the GNU gettext package.
After unpacking the sources, instructions can be found in the directory \texttt{src/po/README.txt}.

It's not too difficult to do the translation.
You don't need to be a programmer.
You must know both English and the language you are translating to, of course.

When you are satisfied with the translation, consider making it available to others.
Upload it at vim-online \url{http://vim.sf.net} or e-mail it to the Vim maintainer <maintainer@vim.org>. Or both. 
\subsection{Language for Menus}
The default menus are in English.
To be able to use your local language, they must be translated.
Normally this is automatically done for you if the environment is set for your language, just like with messages.
You don't need to do anything extra for this.
But it only works if translations for the language are available.

Suppose you are in Germany, with the language set to German, but prefer to use "\texttt{File}" instead of "\texttt{Datei}".
You can switch back to using the English menus this way:

\begin{Verbatim}[samepage=true]
 :set langmenu=none
\end{Verbatim}

It is also possible to specify a language:

\begin{Verbatim}[samepage=true]
 :set langmenu=nl_NL.ISO_8859-1
\end{Verbatim}

Like above, differences between "\texttt{-}" and "\texttt{\_}" matter.
However, upper/lowercase differences are ignored here.

The \texttt{'langmenu'} option must be set before the menus are loaded.
Once the menus have been defined changing \texttt{'langmenu'} has no direct effect.
Therefore, put the command to set \texttt{'langmenu'} in your vimrc file.

If you really want to switch menu language while running Vim, you can do it this way:

\begin{Verbatim}[samepage=true]
 :source $VIMRUNTIME/delmenu.vim
 :set langmenu=de_DE.ISO_8859-1
 :source $VIMRUNTIME/menu.vim
\end{Verbatim}

There is one drawback: All menus that you defined yourself will be gone.
You will need to redefine them as well.

\subsubsection{Do-it-yourself menu translation}
To see which menu translations are available, look in this directory:

\begin{Verbatim}[samepage=true]
    $VIMRUNTIME/lang 
\end{Verbatim}

The files are called \texttt{menu\_{language}.vim}.
If you don't see the language you want to use, you can do your own translations.
The simplest way to do this is by copying one of the existing language files, and change it.

First find out the name of your language with the "\texttt{:language}" command.
Use this name, but with all letters made lowercase.
Then copy the file to your own runtime directory, as found early in \texttt{'runtimepath'}.
For example, for Unix you would do:

\begin{Verbatim}[samepage=true]
 :!cp $VIMRUNTIME/lang/menu_ko_kr.euckr.vim ~/.vim/lang/menu_nl_be.iso_8859-1.vim
\end{Verbatim}

You will find hints for the translation in "\texttt{\$VIMRUNTIME/lang/README.txt}".
\subsection{Using another encoding}
Vim guesses that the files you are going to edit are encoded for your language.
For many European languages this is "\texttt{latin1}".
Then each byte is one character.
That means there are 256 different characters possible.
For Asian languages this is not sufficient.
These mostly use a double-byte encoding, providing for over ten thousand possible characters.
This still isn't enough when a text is to contain several different languages.
This is where Unicode comes in.
It was designed to include all characters used in commonly used languages.
This is the "Super encoding that replaces all others".
But it isn't used that much yet.

Fortunately, Vim supports these three kinds of encodings.
And, with some restrictions, you can use them even when your environment uses another language than the text.

Nevertheless, when you only edit files that are in the encoding of your language, the default should work fine and you don't need to do anything.
The following is only relevant when you want to edit different languages.

\emph{Note}:
Using different encodings only works when Vim was compiled to handle it.
To find out if it works, use the "\texttt{:version}" command and check the output for "\texttt{+multi\_byte}".
If it's there, you are OK.
If you see "\texttt{-multi\_byte}" you will have to find another Vim.

\subsubsection{Using unicode in the gui}
The nice thing about Unicode is that other encodings can be converted to it and back without losing information.
When you make Vim use Unicode internally, you will be able to edit files in any encoding.

Unfortunately, the number of systems supporting Unicode is still limited.
Thus it's unlikely that your language uses it.
You need to tell Vim you want to use Unicode, and how to handle interfacing with the rest of the system.

Let's start with the GUI version of Vim, which is able to display Unicode characters.
This should work:

\begin{Verbatim}[samepage=true]
 :set encoding=utf-8
 :set guifont=-misc-fixed-medium-r-normal--18-120-100-100-c-90-iso10646-1
\end{Verbatim}

The \texttt{'encoding'} option tells Vim the encoding of the characters that you use.
This applies to the text in buffers (files you are editing), registers, Vim script files, etc.
You can regard \texttt{'encoding'} as the setting for the internals of Vim.

This example assumes you have this font on your system.
The name in the example is for the X Window System.
This font is in a package that is used to enhance xterm with Unicode support.
If you don't have this font, you might find it here: \url{http://www.cl.cam.ac.uk/~mgk25/download/ucs-fonts.tar.gz}.

For MS-Windows, some fonts have a limited number of Unicode characters.
Try using the "Courier New" font.
You can use the Edit/Select Font... menu to select and try out the fonts available.
Only fixed-width fonts can be used though.
Example:

\begin{Verbatim}[samepage=true]
 :set guifont=courier_new:h12
\end{Verbatim}

If it doesn't work well, try getting a fontpack.
If Microsoft didn't move it, you can find it here: \url{http://www.microsoft.com/typography/fonts/default.aspx}

Now you have told Vim to use Unicode internally and display text with a Unicode font.
Typed characters still arrive in the encoding of your original language.
This requires converting them to Unicode.
Tell Vim the language from which to convert with the \texttt{'termencoding'} option.
You can do it like this:

\begin{Verbatim}[samepage=true]
 :let &termencoding = &encoding
 :set encoding=utf-8
\end{Verbatim}

This assigns the old value of \texttt{'encoding'} to \texttt{'termencoding'} before setting \texttt{'encoding'} to utf-8.
You will have to try out if this really works for your setup.
It should work especially well when using an input method for an Asian language, and you want to edit Unicode text.

\subsubsection{Using unicode in a unicode terminal}
There are terminals that support Unicode directly.
The standard xterm that comes with XFree86 is one of them.
Let's use that as an example.

First of all, the xterm must have been compiled with Unicode support.
See |\texttt{:h UTF8-xterm}| how to check that and how to compile it when needed.

Start the xterm with the "\texttt{-u8}" argument.
You might also need so specify a font.
Example:

\begin{Verbatim}[samepage=true]
   xterm -u8 -fn -misc-fixed-medium-r-normal--18-120-100-100-c-90-iso10646-1
\end{Verbatim}

Now you can run Vim inside this terminal.
Set \texttt{'encoding'} to "\texttt{utf-8}" as before.
That's all.

\subsubsection{Using unicode in an ordinary terminal}
Suppose you want to work with Unicode files, but don't have a terminal with Unicode support.
You can do this with Vim, although characters that are not supported by the terminal will not be displayed.
The layout of the text will be preserved. 

\begin{Verbatim}[samepage=true]
 :let &termencoding = &encoding
 :set encoding=utf-8
\end{Verbatim}

This is the same as what was used for the GUI.
But it works differently: Vim will convert the displayed text before sending it to the terminal.
That avoids that the display is messed up with strange characters.

For this to work the conversion between \texttt{'termencoding'} and \texttt{'encoding'} must be possible.
Vim will convert from latin1 to Unicode, thus that always works.
For other conversions the |\texttt{:h +iconv}| feature is required.

Try editing a file with Unicode characters in it.
You will notice that Vim will put a question mark (or underscore or some other character) in places where a character should be that the terminal can't display.
Move the cursor to a question mark and use this command:

\begin{Verbatim}[samepage=true]
 ga
\end{Verbatim}

Vim will display a line with the code of the character.
This gives you a hint about what character it is.
You can look it up in a Unicode table.
You could actually view a file that way, if you have lots of time at hand.

\emph{Note}:
Since \texttt{'encoding'} is used for all text inside Vim, changing it makes all non-ASCII text invalid.
You will notice this when using registers and the \texttt{'viminfo'} file (e.g., a remembered search pattern).
It's recommended to set \texttt{'encoding'} in your vimrc file, and leave it alone.
\subsection{Editing files with a different encoding}
Suppose you have setup Vim to use Unicode, and you want to edit a file that is in 16-bit Unicode.
Sounds simple, right?  Well, Vim actually uses utf-8 encoding internally, thus the 16-bit encoding must be converted, since there is a difference between the character set (Unicode) and the encoding (utf-8 or 16-bit).

Vim will try to detect what kind of file you are editing.
It uses the encoding names in the \texttt{'fileencodings'} option.
When using Unicode, the default value is: "\texttt{ucs-bom,utf-8,latin1}".
This means that Vim checks the file to see if it's one of these encodings:

\begin{center} \begin{tabularx}{\textwidth}{l X}
				\texttt{ucs-bom} & File must start with a Byte Order Mark (BOM).  This allows detection of 16-bit, 32-bit and utf-8 Unicode encodings. \\
				\texttt{utf-8} & utf-8 Unicode.  This is rejected when a sequence of bytes is illegal in utf-8. \\
				\texttt{latin1} & The good old 8-bit encoding.  Always works. \\
\end{tabularx} \end{center}
When you start editing that 16-bit Unicode file, and it has a BOM, Vim will detect this and convert the file to utf-8 when reading it.
The \texttt{'fileencoding'} option (without s at the end) is set to the detected value.
In this case it is "\texttt{utf-16le}".
That means it's Unicode, 16-bit and little-endian.
This file format is common on MS-Windows (e.g., for registry files).

When writing the file, Vim will compare \texttt{'fileencoding'} with \texttt{'encoding'}.
If they are different, the text will be converted.

An empty value for \texttt{'fileencoding'} means that no conversion is to be done.
Thus the text is assumed to be encoded with \texttt{'encoding'}.

If the default \texttt{'fileencodings'} value is not good for you, set it to the encodings you want Vim to try.
Only when a value is found to be invalid will the next one be used.
Putting "\texttt{latin1}" first doesn't work, because it is never illegal.
An example, to fall back to Japanese when the file doesn't have a BOM and isn't utf-8:

\begin{Verbatim}[samepage=true]
 :set fileencodings=ucs-bom,utf-8,sjis
\end{Verbatim}

See |\texttt{:h encoding-values}| for suggested values.
Other values may work as well.
This depends on the conversion available.

\subsubsection{Forcing an encoding}
If the automatic detection doesn't work you must tell Vim what encoding the file is.
Example:

\begin{Verbatim}[samepage=true]
 :edit ++enc=koi8-r russian.txt
\end{Verbatim}

The "\texttt{++enc}" part specifies the name of the encoding to be used for this file only.
Vim will convert the file from the specified encoding, Russian in this example, to \texttt{'encoding'}.
\texttt{'fileencoding'} will also be set to the specified encoding, so that the reverse conversion can be done when writing the file.

The same argument can be used when writing the file.
This way you can actually use Vim to convert a file.
Example:

\begin{Verbatim}[samepage=true]
 :write ++enc=utf-8 russian.txt
\end{Verbatim}
 
\emph{Note}: Conversion may result in lost characters.
Conversion from an encoding to Unicode and back is mostly free of this problem, unless there are illegal characters.
Conversion from Unicode to other encodings often loses information when there was more than one language in the file.
\subsection{Entering language text}
\label{Entering language text}
Computer keyboards don't have much more than a hundred keys.
Some languages have thousands of characters, Unicode has ten thousands.
So how do you type these characters?

First of all, when you don't use too many of the special characters, you can use digraphs.
This was already explained in |\hyperref[Digraphs]{\texttt{Digraphs}}|.

When you use a language that uses many more characters than keys on your keyboard, you will want to use an Input Method (IM).
This requires learning the translation from typed keys to resulting character.
When you need an IM you probably already have one on your system.
It should work with Vim like with other programs.
For details see |\texttt{:h mbyte-XIM}| for the X Window system and |\texttt{:h mbyte-IME}| for MS-Windows.

\subsubsection{Keymaps}
For some languages the character set is different from latin, but uses a similar number of characters.
It's possible to map keys to characters.
Vim uses keymaps for this.

Suppose you want to type Hebrew.
You can load the keymap like this:

\begin{Verbatim}[samepage=true]
 :set keymap=hebrew
\end{Verbatim}

Vim will try to find a keymap file for you.
This depends on the value of \texttt{'encoding'}.
If no matching file was found, you will get an error message.

Now you can type Hebrew in Insert mode.
In Normal mode, and when typing a "\texttt{:}" command, Vim automatically switches to English.
You can use this command to switch between Hebrew and English:

\begin{Verbatim}[samepage=true]
 CTRL-^
\end{Verbatim}

This only works in Insert mode and Command-line mode.
In Normal mode it does something completely different (jumps to alternate file).

The usage of the keymap is indicated in the mode message, if you have the \texttt{'showmode'} option set.
In the GUI Vim will indicate the usage of keymaps with a different cursor color.

You can also change the usage of the keymap with the \texttt{'iminsert'} and \texttt{'imsearch'} options.

To see the list of mappings, use this command:

\begin{Verbatim}[samepage=true]
 :lmap
\end{Verbatim}

To find out which keymap files are available, in the GUI you can use the Edit/Keymap menu.
Otherwise you can use this command:

\begin{Verbatim}[samepage=true]
 :echo globpath(&rtp, "keymap/*.vim")
\end{Verbatim}

\subsubsection{Do-it-yourself keymaps}
You can create your own keymap file.
It's not very difficult.
Start with a keymap file that is similar to the language you want to use.
Copy it to the "\texttt{keymap}" directory in your runtime directory.
For example, for Unix, you would use the directory "\texttt{~/.vim/keymap}".

The name of the keymap file must look like this:

\begin{Verbatim}[samepage=true]
    keymap/{name}.vim 
\end{Verbatim}
or
\begin{Verbatim}[samepage=true]
    keymap/{name}_{encoding}.vim 
\end{Verbatim}

\texttt{{name}} is the name of the keymap.
Chose a name that is obvious, but different from existing keymaps (unless you want to replace an existing keymap file).
\texttt{{name}} cannot contain an underscore.
Optionally, add the encoding used after an underscore.
Examples:

\begin{Verbatim}[samepage=true]
    keymap/hebrew.vim 
    keymap/hebrew_utf-8.vim 
\end{Verbatim}

The contents of the file should be self-explanatory.
Look at a few of the keymaps that are distributed with Vim.
For the details, see |\texttt{:h mbyte-keymap}|.

\subsubsection{Last resort}
If all other methods fail, you can enter any character with CTRL-V:

\begin{center} \begin{tabular}{l l l}
				encoding & type & range \\
				8-bit & CTRL-V 123 & decimal 0-255 \\
				8-bit & CTRL-V x a1 & hexadecimal 00-ff \\
				16-bit & CTRL-V u 013b & hexadecimal 0000-ffff \\
				31-bit & CTRL-V U 001303a4 & hexadecimal 00000000-7fffffff \\
\end{tabular} \end{center}

Don't type the spaces.
See |\texttt{:h i\_CTRL-V\_digit}| for the details.
\clearpage
