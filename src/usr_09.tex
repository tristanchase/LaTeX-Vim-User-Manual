\section{09. Using the GUI}
Vim works in an ordinary terminal.  GVim can do the same things and a few
more.  The GUI offers menus, a toolbar, scrollbars and other items.  This
chapter is about these extra things that the GUI offers.
%\localtableofcontentswithrelativedepth{+1} % I couldn't get this to work TMC
\etocsetnexttocdepth{2} % This produces the same result TMC
\etocsettocstyle{\subsection*{Contents}}{}
\localtableofcontents
\subsection{Parts of the GUI}

You might have an icon on your desktop that starts gVim.
Otherwise, one of these commands should do it:

 \begin{Verbatim}[samepage=true]
 gvim file.txt
 vim -g file.txt
 \end{Verbatim}

If this doesn't work you don't have a version of Vim with GUI support.
You will have to install one first.

Vim will open a window and display ``\texttt{file.txt}" in it.
What the window looks like depends on the version of Vim.
It should resemble the following picture (for as far as this can be shown in ASCII!).

\begin{Verbatim}[samepage=true]
    +----------------------------------------------------+
    | file.txt + (~/dir) - VIM                         X | <- window title
    +----------------------------------------------------+
    | File Edit  Tools  Syntax  Buffers  Window  Help    | <- menubar
    +----------------------------------------------------+
    | aaa  bbb  ccc  ddd  eee  fff ggg  hhh  iii  jjj    | <- toolbar
    | aaa  bbb  ccc  ddd  eee  fff ggg  hhh  iii  jjj    |
    +----------------------------------------------------+
    | file text                                      | ^ |
    | ~                                              | # |
    | ~                                              | # | <- scrollbar
    | ~                                              | # |
    | ~                                              | # |
    | ~                                              | # |
    |                                                | V |
    +----------------------------------------------------+
\end{Verbatim}

The largest space is occupied by the file text.
This shows the file in the same way as in a terminal.
With some different colors and another font perhaps.
\subsubsection{The window title}
At the very top is the window title.  This is drawn by your window system.
Vim will set the title to show the name of the current file.  First comes the
name of the file.  Then some special characters and the directory of the file
in parens.  These special character can be present:

\begin{center}\begin{longtable}{c l}
				\texttt{-} & The file cannot be modified (e.g., a help file) \\
				\texttt{+} & The file contains changes \\
				\texttt{=} & The file is read-only \\
				\texttt{=+} & The file is read-only, contains changes anyway \\
\end{longtable}\end{center}
If nothing is shown you have an ordinary, unchanged file.
\subsubsection{The menubar}
You know how menus work, right?  Vim has the usual items, plus a few more.
Browse them to get an idea of what you can use them for.
A relevant submenu is Edit/Global Settings.
You will find these entries:

\begin{center}\begin{longtable}{c l}
				Toggle Toolbar & make the toolbar appear/disappear \\
				Toggle Bottom Scrollbar & make a scrollbar appear/disappear at the bottom \\
				Toggle Left Scrollbar & make a scrollbar appear/disappear at the left \\
				Toggle Right Scrollbar & make a scrollbar appear/disappear at the right \\
\end{longtable}\end{center}

On most systems you can tear-off the menus.
Select the top item of the menu, the one that looks like a dashed line.
You will get a separate window with the items of the menu.
It will hang around until you close the window.

\subsubsection{The toolbar}
This contains icons for the most often used actions.
Hopefully the icons are self-explanatory.
There are tooltips to get an extra hint (move the mouse pointer to the icon without clicking and don't move it for a second).

The ``Edit/Global Settings/Toggle Toolbar" menu item can be used to make the toolbar disappear.
If you never want a toolbar, use this command in your vimrc file:

 \begin{Verbatim}[samepage=true]
 :set guioptions-=T
 \end{Verbatim}

This removes the \texttt{'T'} flag from the \texttt{'guioptions'} option.
Other parts of the GUI can also be enabled or disabled with this option.
See the help for it.

\subsubsection{The scrollbars}
By default there is one scrollbar on the right.
It does the obvious thing.
When you split the window, each window will get its own scrollbar.

You can make a horizontal scrollbar appear with the menu item Edit/Global Settings/Toggle Bottom Scrollbar.
This is useful in diff mode, or when the \texttt{'wrap'} option has been reset (more about that later).

When there are vertically split windows, only the windows on the right side will have a scrollbar.
However, when you move the cursor to a window on the left, it will be this one the that scrollbar controls.
This takes a bit of time to get used to.

When you work with vertically split windows, consider adding a scrollbar on the left.
This can be done with a menu item, or with the \texttt{'guioptions'} option:

 \begin{Verbatim}[samepage=true]
 :set guioptions+=l
 \end{Verbatim}

This adds the \texttt{'l'} flag to \texttt{'guioptions'}.
\subsection{Using the mouse}
Standards are wonderful.
In Microsoft Windows, you can use the mouse to select text in a standard manner.
The X Window system also has a standard system for using the mouse.
Unfortunately, these two standards are not the same.

Fortunately, you can customize Vim.
You can make the behavior of the mouse work like an X Window system mouse or a Microsoft Windows mouse.
The following command makes the mouse behave like an X Window mouse:

 \begin{Verbatim}[samepage=true]
 :behave xterm
 \end{Verbatim}

The following command makes the mouse work like a Microsoft Windows mouse:

 \begin{Verbatim}[samepage=true]
 :behave mswin
 \end{Verbatim}

The default behavior of the mouse on UNIX systems is xterm.
The default behavior on a Microsoft Windows system is selected during the installation process.
For details about what the two behaviors are, see |\texttt{:h :behave}|.
Here follows a summary.
\subsubsection{Xterm mouse behavior}

\begin{center}\begin{longtable}{c l}
				Left mouse click & position the cursor \\
				Left mouse drag & select text in Visual mode \\
				Middle mouse click & paste text from the clipboard \\
				Right mouse click & extend the selected text until the mouse pointer \\
\end{longtable}\end{center}

\subsubsection{Mswin mouse behavior}

\begin{center}\begin{longtable}{l l}
				Left mouse click & position the cursor \\
				Left mouse drag & select text in Select mode (see |\hyperref[Select mode]{\texttt{Select mode}}|) \\
				Left mouse click, with Shift & extend the selected text until the mouse pointer \\
				Middle mouse click & paste text from the clipboard \\
				Right mouse click & display a pop-up menu \\
\end{longtable}\end{center}

The mouse can be further tuned.
Check out these options if you want to change the way how the mouse works:

\begin{center}\begin{longtable}{l l}
				\texttt{'mouse'} & in which mode the mouse is used by Vim \\
				\texttt{'mousemodel'} & what effect a mouse click has \\
				\texttt{'mousetime'} & time between clicks for a double-click \\
				\texttt{'mousehide'} & hide the mouse while typing \\
				\texttt{'selectmode'} & whether the mouse starts Visual or Select mode \\
\end{longtable}\end{center}
\subsection{The clipboard}
In section |\hyperref[Using the clipboard]{\texttt{Using the clipboard}}| the basic use of the clipboard was explained.
There is one essential thing to explain about X-windows: There are actually two places to exchange text between programs.
MS-Windows doesn't have this.

In X-Windows there is the ``current selection".
This is the text that is currently highlighted.
In Vim this is the Visual area (this assumes you are using the default option settings).
You can paste this selection in another application without any further action.

For example, in this text select a few words with the mouse.
Vim will switch to Visual mode and highlight the text.
Now start another gVim, without a file name argument, so that it displays an empty window.
Click the middle mouse button.
The selected text will be inserted.

The ``current selection" will only remain valid until some other text is selected.
After doing the paste in the other gVim, now select some characters in that window.
You will notice that the words that were previously selected in the other gVim window are displayed differently.
This means that it no longer is the current selection.

You don't need to select text with the mouse, using the keyboard commands for Visual mode works just as well.
\subsubsection{The real clipboard}
Now for the other place with which text can be exchanged.
We call this the ``real clipboard", to avoid confusion.
Often both the ``current selection" and the ``real clipboard" are called clipboard, you'll have to get used to that.

To put text on the real clipboard, select a few different words in one of the gVims you have running.
Then use the Edit/Copy menu entry.
Now the text has been copied to the real clipboard.
You can't see this, unless you have some application that shows the clipboard contents (e.g., KDE's klipper).

Now select the other gVim, position the cursor somewhere and use the Edit/Paste menu.
You will see the text from the real clipboard is inserted.
\subsubsection{Using both}
This use of both the ``current selection" and the ``real clipboard" might sound a bit confusing.
But it is very useful.
Let's show this with an example.
Use one gVim with a text file and perform these actions:

\begin{itemize}
				\item Select two words in Visual mode.
				\item Use the Edit/Copy menu to get these words onto the clipboard.
				\item Select one other word in Visual mode.
				\item Use the Edit/Paste menu item.
								What will happen is that the single selected word is replaced with the two words from the clipboard.
				\item Move the mouse pointer somewhere else and click the middle button.
								You will see that the word you just overwrote with the clipboard is inserted here.
\end{itemize}

If you use the ``current selection" and the ``real clipboard" with care, you can do a lot of useful editing with them.
\subsubsection{Using the keyboard}

If you don't like using the mouse, you can access the current selection and the real clipboard with two registers.
The \texttt{"*} register is for the current selection.

To make text become the current selection, use Visual mode.
For example, to select a whole line just press ``\texttt{V}".

To insert the current selection before the cursor:

 \begin{Verbatim}[samepage=true]
 "*P
 \end{Verbatim}

Notice the uppercase ``\texttt{P}".
The lowercase ``\texttt{p}" puts the text after the cursor.

The \texttt{"+} register is used for the real clipboard.
For example, to copy the text from the cursor position until the end of the line to the clipboard:

 \begin{Verbatim}[samepage=true]
 "+y$
 \end{Verbatim}

Remember, ``\texttt{y}" is yank, which is Vim's copy command.

To insert the contents of the real clipboard before the cursor:

 \begin{Verbatim}[samepage=true]
 "+P
 \end{Verbatim}

It's the same as for the current selection, but uses the plus (\texttt{+}) register instead of the star (\texttt{*}) register.
\subsection{Select mode}
\label{Select mode}
And now something that is used more often on MS-Windows than on X-Windows.
But both can do it.
You already know about Visual mode.
Select mode is like Visual mode, because it is also used to select text.
But there is an obvious difference: When typing text, the selected text is deleted and the typed text replaces it.

To start working with Select mode, you must first enable it (for MS-Windows it is probably already enabled, but you can do this anyway):

 \begin{Verbatim}[samepage=true]
 :set selectmode+=mouse
 \end{Verbatim}

Now use the mouse to select some text.
It is highlighted like in Visual mode.
Now press a letter.
The selected text is deleted, and the single letter replaces it.
You are in Insert mode now, thus you can continue typing.

Since typing normal text causes the selected text to be deleted, you can not use the normal movement commands ``hjkl", ``w", etc.
Instead, use the shifted function keys.
<S-Left> (shifted cursor left key) moves the cursor left.
The selected text is changed like in Visual mode.
The other shifted cursor keys do what you expect.
<S-End> and <S-Home> also work.

You can tune the way Select mode works with the \texttt{'selectmode'} option.
\clearpage
