\section{24. Inserting quickly}
When entering text, Vim offers various ways to reduce the number of keystrokes and avoid typing mistakes.
Use Insert mode completion to repeat previously typed words.
Abbreviate long words to short ones.
Type characters that aren't on your keyboard.
\localtableofcontentswithrelativedepth{+1}
\subsection{Making corrections}
The <BS> key was already mentioned.
It deletes the character just before the cursor.
The <Del> key does the same for the character under (after) the cursor.

\begin{Verbatim}[samepage=true]

When you typed a whole word wrong, use CTRL-W:

    The horse had fallen to the sky 
                                   CTRL-W
    The horse had fallen to the 
\end{Verbatim}

If you really messed up a line and want to start over, use CTRL-U to delete it.
This keeps the text after the cursor and the indent.
Only the text from the first non-blank to the cursor is deleted.
With the cursor on the "\texttt{f}" of "\texttt{fallen}" in the next line pressing CTRL-U does this:

\begin{Verbatim}[samepage=true]
    The horse had fallen to the 
                  CTRL-U
    fallen to the 
\end{Verbatim}

When you spot a mistake a few words back, you need to move the cursor there to correct it.
For example, you typed this:

\begin{Verbatim}[samepage=true]
    The horse had follen to the ground 
\end{Verbatim}

You need to change "\texttt{follen}" to "\texttt{fallen}".
With the cursor at the end, you would type this to correct it:

\begin{Verbatim}[samepage=true]
                               <Esc>4blraA

    get out of Insert mode     <Esc>
    four words back                 4b
    move on top of the "o"            l
    replace with "a"                   ra
    restart Insert mode                  A
\end{Verbatim}

Another way to do this:

\begin{Verbatim}[samepage=true]
     <C-Left><C-Left><C-Left><C-Left><Right><Del>a<End>

    four words back          <C-Left><C-Left><C-Left><C-Left>
    move on top of the "o"            <Right>
    delete the "o"                       <Del>
    insert an "a"                             a
    go to end of the line                      <End>
\end{Verbatim}

This uses special keys to move around, while remaining in Insert mode.
This resembles what you would do in a modeless editor.
It's easier to remember, but takes more time (you have to move your hand from the letters to the cursor keys, and the <End> key is hard to press without looking at the keyboard).
These special keys are most useful when writing a mapping that doesn't leave Insert mode.
The extra typing doesn't matter then.
An overview of the keys you can use in Insert mode:

\begin{center} \begin{tabular}{l l}
				<C-Home> & to start of the file \\
				<PageUp> & a whole screenful up \\
				<Home> & to start of line \\
				<S-Left> & one word left \\
				<C-Left> & one word left \\
				<S-Right> & one word right \\
				<C-Right> & one word right \\
				<End> & to end of the line \\
				<PageDown> & a whole screenful down \\
				<C-End> & to end of the file
\end{tabular} \end{center}

There are a few more, see |\texttt{:h ins-special-special}|.
\subsection{Showing matches}
When you type a \texttt{)} it would be nice to see with which \texttt{(} it matches.
To make Vim do that use this command:

\begin{Verbatim}[samepage=true]
 :set showmatch
\end{Verbatim}

When you now type a text like "\texttt{(example)}", as soon as you type the \texttt{)} Vim will briefly move the cursor to the matching \texttt{(}, keep it there for half a second, and move back to where you were typing.
In case there is no matching \texttt{(}, Vim will beep.
Then you know that you might have forgotten the \texttt{(} somewhere, or typed a \texttt{)} too many.
The match will also be shown for \texttt{[]} and \texttt{{}} pairs.
You don't have to wait with typing the next character, as soon as Vim sees it the cursor will move back and inserting continues as before.
You can change the time Vim waits with the \texttt{'matchtime'} option.
For example, to make Vim wait one and a half second:

\begin{Verbatim}[samepage=true]
 :set matchtime=15
\end{Verbatim}

The time is specified in tenths of a second.
\subsection{Completion}
Vim can automatically complete words on insertion.
You type the first part of a word, press CTRL-P, and Vim guesses the rest.

Suppose, for example, that you are creating a C program and want to type in the following:

\begin{Verbatim}[samepage=true]
    total = ch_array[0] + ch_array[1] + ch_array[2]; 
\end{Verbatim}

You start by entering the following:

\begin{Verbatim}[samepage=true]
    total = ch_array[0] + ch_ 
\end{Verbatim}

At this point, you tell Vim to complete the word using the command CTRL-P.
Vim searches for a word that starts with what's in front of the cursor.
In this case, it is "\texttt{ch\_}", which matches with the word \texttt{ch\_array}.
So typing CTRL-P gives you the following:

\begin{Verbatim}[samepage=true]
    total = ch_array[0] + ch_array 
\end{Verbatim}

After a little more typing, you get this (ending in a space):

\begin{Verbatim}[samepage=true]
    total = ch_array[0] + ch_array[1] +  
\end{Verbatim}

If you now type CTRL-P Vim will search again for a word that completes the word before the cursor.
Since there is nothing in front of the cursor, it finds the first word backwards, which is "\texttt{ch\_array}".
Typing CTRL-P again gives you the next word that matches, in this case "\texttt{total}".
A third CTRL-P searches further back.
If there is nothing else, it causes the editor to run out of words, so it returns to the original text, which is nothing.
A fourth CTRL-P causes the editor to start over again with "\texttt{ch\_array}".

To search forward, use CTRL-N.
Since the search wraps around the end of the file, CTRL-N and CTRL-P will find the same matches, but in a different sequence.
Hint: CTRL-N is Next-match and CTRL-P is Previous-match.

The Vim editor goes through a lot of effort to find words to complete.
By default, it searches the following places:
\begin{enumerate}
				\item Current file
				\item Files in other windows
				\item Other loaded files (hidden buffers)
				\item Files which are not loaded (inactive buffers)
				\item Tag files
				\item All files \texttt{\#included} by the current file
\end{enumerate}
\subsubsection{Options}
You can customize the search order with the \texttt{'complete'} option.

The \texttt{'ignorecase'} option is used.
When it is set, case differences are ignored when searching for matches.

A special option for completion is \texttt{'infercase'}.
This is useful to find matches while ignoring case (\texttt{'ignorecase'} must be set) but still using the case of the word typed so far.
Thus if you type "\texttt{For}" and Vim finds a match "fortunately", it will result in "Fortunately".
\subsubsection{Completing specific items}
If you know what you are looking for, you can use these commands to complete with a certain type of item:
\begin{center} \begin{tabular}{l l}
				CTRL-X CTRL-F & file names \\
				CTRL-X CTRL-L & whole lines \\
				CTRL-X CTRL-D & macro definitions (also in included files) \\
				CTRL-X CTRL-I & current and included files \\
				CTRL-X CTRL-K & words from a dictionary \\
				CTRL-X CTRL-T & words from a thesaurus \\
				CTRL-X CTRL-] & tags \\
				CTRL-X CTRL-V & Vim command line \\
\end{tabular} \end{center}
After each of them CTRL-N can be used to find the next match, CTRL-P to find the previous match.

More information for each of these commands here: |\texttt{:h ins-completion}|.
\subsubsection{Completing file names}
Let's take CTRL-X CTRL-F as an example.
This will find file names.
It scans the current directory for files and displays each one that matches the word in front of the cursor.

Suppose, for example, that you have the following files in the current directory:

\begin{Verbatim}[samepage=true]
    main.c  sub_count.c  sub_done.c  sub_exit.c
\end{Verbatim}

Now enter Insert mode and start typing:

\begin{Verbatim}[samepage=true]
    The exit code is in the file sub 
\end{Verbatim}

At this point, you enter the command CTRL-X CTRL-F.
Vim now completes the current word "\texttt{sub}" by looking at the files in the current directory.
The first match is \texttt{sub\_count.c}.
This is not the one you want, so you match the next file by typing CTRL-N.
This match is \texttt{sub\_done.c}.
Typing CTRL-N again takes you to \texttt{sub\_exit.c}.
The results:

\begin{Verbatim}[samepage=true]
    The exit code is in the file sub_exit.c 
\end{Verbatim}

If the file name starts with \texttt{/} (Unix) or \texttt{C:\textbackslash{}} (MS-Windows) you can find all files in the file system.
For example, type "\texttt{/u}" and CTRL-X CTRL-F.
This will match "\texttt{/usr}" (this is on Unix):

\begin{Verbatim}[samepage=true]
    the file is found in /usr/ 
\end{Verbatim}

If you now press CTRL-N you go back to "\texttt{/u}".
Instead, to accept the "\texttt{/usr/}" and go one directory level deeper, use CTRL-X CTRL-F again:

\begin{Verbatim}[samepage=true]
    the file is found in /usr/X11R6/ 
\end{Verbatim}

The results depend on what is found in your file system, of course.
The matches are sorted alphabetically.
\subsubsection{Completing in source code}
Source code files are well structured.
That makes it possible to do completion in an intelligent way.
In Vim this is called Omni completion.
In some other editors it's called intellisense, but that is a trademark.

The key to Omni completion is CTRL-X CTRL-O.
Obviously the O stands for Omni here, so that you can remember it easier.
Let's use an example for editing C source:

\begin{Verbatim}[samepage=true]
    { 
        struct foo *p; 
        p-> 
\end{Verbatim}

The cursor is after "\texttt{p->}".
Now type CTRL-X CTRL-O.
Vim will offer you a list of alternatives, which are the items that "\texttt{struct foo}" contains.
That is quite different from using CTRL-P, which would complete any word, while only members of "\texttt{struct foo}" are valid here.

For Omni completion to work you may need to do some setup.
At least make sure filetype plugins are enabled.
Your vimrc file should contain a line like this:

\begin{Verbatim}[samepage=true]
 filetype plugin on
\end{Verbatim}
 
Or:

\begin{Verbatim}[samepage=true]
 filetype plugin indent on
\end{Verbatim}

For C code you need to create a tags file and set the \texttt{'tags'} option.
That is explained |\texttt{:h ft-c-omni}|.
For other filetypes you may need to do something similar, look below |\texttt{:h compl-omni-filetypes}|.
It only works for specific filetypes.
Check the value of the \texttt{'omnifunc'} option to find out if it would work.
\subsection{Repeating an insert}
If you press CTRL-A, the editor inserts the text you typed the last time you were in Insert mode.

Assume, for example, that you have a file that begins with the following:

\begin{Verbatim}[samepage=true]
    "file.h" 
    /* Main program begins */ 
\end{Verbatim}

You edit this file by inserting "\texttt{\#include }" at the beginning of the first line:

\begin{Verbatim}[samepage=true]
    #include "file.h" 
    /* Main program begins */ 
\end{Verbatim}

You go down to the beginning of the next line using the commands "\texttt{j\^{}}".
You now start to insert a new "\texttt{\#include}" line.
So you type:

\begin{Verbatim}[samepage=true]
 i CTRL-A
\end{Verbatim}

The result is as follows:

\begin{Verbatim}[samepage=true]
    #include "file.h" 
    #include /* Main program begins */ 
\end{Verbatim}

The "\texttt{\#include }" was inserted because CTRL-A inserts the text of the previous insert.
Now you type  "\texttt{main.h}"<Enter>  to finish the line:


\begin{Verbatim}[samepage=true]
    #include "file.h" 
    #include "main.h" 
    /* Main program begins */ 
\end{Verbatim}

The CTRL-@ command does a CTRL-A and then exits Insert mode.
That's a quick way of doing exactly the same insertion again.
\subsection{Copying from another line}
The CTRL-Y command inserts the character above the cursor.
This is useful when you are duplicating a previous line.
For example, you have this line of C code:

\begin{Verbatim}[samepage=true]
    b_array[i]->s_next = a_array[i]->s_next; 
\end{Verbatim}

Now you need to type the same line, but with "\texttt{s\_prev}" instead of "\texttt{s\_next}".
Start the new line, and press CTRL-Y 14 times, until you are at the "\texttt{n}" of "\texttt{next}":

\begin{Verbatim}[samepage=true]
    b_array[i]->s_next = a_array[i]->s_next; 
    b_array[i]->s_ 
\end{Verbatim}

Now you type "\texttt{prev}":

\begin{Verbatim}[samepage=true]
    b_array[i]->s_next = a_array[i]->s_next; 
    b_array[i]->s_prev 
\end{Verbatim}

Continue pressing CTRL-Y until the following "\texttt{next}":

\begin{Verbatim}[samepage=true]
    b_array[i]->s_next = a_array[i]->s_next; 
    b_array[i]->s_prev = a_array[i]->s_ 
\end{Verbatim}

Now type "\texttt{prev;}" to finish it off.

The CTRL-E command acts like CTRL-Y except it inserts the character below the cursor.
\subsection{Inserting a register}
The command CTRL-R \texttt{{register}} inserts the contents of the register.
This is useful to avoid having to type a long word.
For example, you need to type this:

\begin{Verbatim}[samepage=true]
    r = VeryLongFunction(a) + VeryLongFunction(b) + VeryLongFunction(c) 
\end{Verbatim}

The function name is defined in a different file.
Edit that file and move the cursor on top of the function name there, and yank it into register v:

\begin{Verbatim}[samepage=true]
 "vyiw
\end{Verbatim}

\texttt{"v} is the register specification, "\texttt{yiw}" is yank-inner-word.
Now edit the file where the new line is to be inserted, and type the first letters:

\begin{Verbatim}[samepage=true]
    r = 
\end{Verbatim}

Now use CTRL-R v to insert the function name:

\begin{Verbatim}[samepage=true]
    r = VeryLongFunction 
\end{Verbatim}

You continue to type the characters in between the function name, and use CTRL-R v two times more.

You could have done the same with completion.
Using a register is useful when there are many words that start with the same characters.

If the register contains characters such as <BS> or other special characters, they are interpreted as if they had been typed from the keyboard.
If you do not want this to happen (you really want the <BS> to be inserted in the text), use the command CTRL-R CTRL-R \texttt{{register}.}
\subsection{Abbreviations}
An abbreviation is a short word that takes the place of a long one.
For example, "ad" stands for "advertisement".
Vim enables you to type an abbreviation and then will automatically expand it for you.

To tell Vim to expand "ad" into "advertisement" every time you insert it, use the following command:

\begin{Verbatim}[samepage=true]
 :iabbrev ad advertisement
\end{Verbatim}

Now, when you type "ad", the whole word "advertisement" will be inserted into the text.
This is triggered by typing a character that can't be part of a word, for example a space:

\begin{center} \begin{tabular}{l l}
				What Is Entered & What You See \\
				I saw the a & I saw the a \\
				I saw the ad & I saw the ad \\
				I saw the ad<Space> & I saw the advertisement<Space> \\
\end{tabular} \end{center}
The expansion doesn't happen when typing just "ad".
That allows you to type a word like "add", which will not get expanded.
Only whole words are checked for abbreviations.
\subsubsection{Abbreviating several words}
It is possible to define an abbreviation that results in multiple words.
For example, to define "JB" as "Jack Benny", use the following command:

\begin{Verbatim}[samepage=true]
 :iabbrev JB Jack Benny
\end{Verbatim}

As a programmer, I use two rather unusual abbreviations:

\begin{Verbatim}[samepage=true]
 :iabbrev #b /****************************************
 :iabbrev #e <Space>****************************************/
\end{Verbatim}

These are used for creating boxed comments.
The comment starts with \texttt{\#b}, which draws the top line.
I then type the comment text and use \texttt{\#e} to draw the bottom line.

Notice that the \texttt{\#e} abbreviation begins with a space.
In other words, the first two characters are space-star.
Usually Vim ignores spaces between the abbreviation and the expansion.
To avoid that problem, I spell space as seven characters: <, S, p, a, c, e, >.

\emph{Note}: "\texttt{:iabbrev}" is a long word to type.
"\texttt{:iab}" works just as well.
That's abbreviating the abbreviate command!
\subsubsection{Fixing typing mistakes}
It's very common to make the same typing mistake every time.
For example, typing "teh" instead of "the".
You can fix this with an abbreviation:

\begin{Verbatim}[samepage=true]
 :abbreviate teh the
\end{Verbatim}

You can add a whole list of these.
Add one each time you discover a common mistake.
\subsubsection{Listing abbreviations}
The "\texttt{:abbreviate}" command lists the abbreviations:

\begin{Verbatim}[samepage=true]
    :abbreviate
    i  #e         ****************************************/
    i  #b        /****************************************
    i  JB        Jack Benny
    i  ad        advertisement
    !  teh       the
\end{Verbatim}

The "\texttt{i}" in the first column indicates Insert mode.
These abbreviations are only active in Insert mode.
Other possible characters are:

\begin{center} \begin{tabular}{c l l}
				\texttt{c} & Command-line mode & \texttt{:cabbrev}\\
				\texttt{!} & both Insert and Command-line mode & \texttt{:abbreviate}\\
\end{tabular} \end{center}

Since abbreviations are not often useful in Command-line mode, you will mostly use the "\texttt{:iabbrev}" command.
That avoids, for example, that "\texttt{ad}" gets expanded when typing a command like:

\begin{Verbatim}[samepage=true]
 :edit ad
\end{Verbatim}
\subsubsection{Deleting abbreviations}
To get rid of an abbreviation, use the "\texttt{:unabbreviate}" command.
Suppose you have the following abbreviation:

\begin{Verbatim}[samepage=true]
 :abbreviate @f fresh
\end{Verbatim}

You can remove it with this command:

\begin{Verbatim}[samepage=true]
 :unabbreviate @f
\end{Verbatim}

While you type this, you will notice that @f is expanded to "fresh".
Don't worry about this, Vim understands it anyway (except when you have an abbreviation for "fresh", but that's very unlikely).

To remove all the abbreviations:

\begin{Verbatim}[samepage=true]
 :abclear
\end{Verbatim}

"\texttt{:unabbreviate}" and "\texttt{:abclear}" also come in the variants for Insert mode ("\texttt{:iunabbreviate}" and "\texttt{:iabclear}") and Command-line mode ("\texttt{:cunabbreviate}" and "\texttt{:cabclear}").
\subsubsection{Remapping abbreviations}
There is one thing to watch out for when defining an abbreviation: The resulting string should not be mapped.
For example:

\begin{Verbatim}[samepage=true]
 :abbreviate @a adder
 :imap dd disk-door
\end{Verbatim}

When you now type \texttt{@a}, you will get "adisk-doorer".
That's not what you want.
To avoid this, use the "\texttt{:noreabbrev}" command.
It does the same as "\texttt{:abbreviate}", but avoids that the resulting string is used for mappings:

\begin{Verbatim}[samepage=true]
 :noreabbrev @a adder
\end{Verbatim}

Fortunately, it's unlikely that the result of an abbreviation is mapped.
\subsection{Entering special characters}
The CTRL-V command is used to insert the next character literally.
In other words, any special meaning the character has, it will be ignored.
For example:

\begin{Verbatim}[samepage=true]
 CTRL-V <Esc>
\end{Verbatim}

Inserts an escape character.
Thus you don't leave Insert mode.
(Don't type the space after CTRL-V, it's only to make this easier to read).

\emph{Note}: On MS-Windows CTRL-V is used to paste text.
Use CTRL-Q instead of CTRL-V.
On Unix, on the other hand, CTRL-Q does not work on some terminals, because it has a special meaning.

You can also use the command CTRL-V \texttt{{digits}} to insert a character with the decimal number \texttt{{digits}.}
For example, the character number 127 is the <Del> character (but not necessarily the <Del> key!).
To insert <Del> type:

\begin{Verbatim}[samepage=true]
 CTRL-V 127
\end{Verbatim}

You can enter characters up to 255 this way.
When you type fewer than two digits, a non-digit will terminate the command.
To avoid the need of typing a non-digit, prepend one or two zeros to make three digits.

All the next commands insert a <Tab> and then a dot:

\begin{Verbatim}[samepage=true]
    CTRL-V 9.
    CTRL-V 09.
    CTRL-V 009.
\end{Verbatim}

To enter a character in hexadecimal, use an "\texttt{x}" after the CTRL-V:

\begin{Verbatim}[samepage=true]
 CTRL-V x7f
\end{Verbatim}

This also goes up to character 255 (CTRL-V xff).
You can use "\texttt{o}" to type a character as an octal number and two more methods allow you to type up to a 16 bit and a 32 bit number (e.g., for a Unicode character):

\begin{Verbatim}[samepage=true]
 CTRL-V o123
 CTRL-V u1234
 CTRL-V U12345678
\end{Verbatim}
\subsection{Digraphs}
\label{Digraphs}
Some characters are not on the keyboard.
For example, the copyright character (©).
To type these characters in Vim, you use digraphs, where two characters represent one.
To enter a copyright, for example, you press three keys:

\begin{Verbatim}[samepage=true]
 CTRL-K Co
\end{Verbatim}

To find out what digraphs are available, use the following command:

\begin{Verbatim}[samepage=true]
 :digraphs
\end{Verbatim}

Vim will display the digraph table.
Here are three lines of it:

\begin{Verbatim}[samepage=true]
  AC ~_ 159  NS |  160  !I ¡  161  Ct ¢  162  Pd £  163  Cu ¤  164  Ye ¥  165
  BB ¦  166  SE §  167  ': ¨  168  Co ©  169  -a ª  170  << «  171  NO ¬  172
  --    173  Rg ®  174  'm ¯  175  DG °  176  +- ±  177  2S ²  178  3S ³  179 
\end{Verbatim}
 
This shows, for example, that the digraph you get by typing CTRL-K Pd is the character (£).
This is character number 163 (decimal).

Pd is short for Pound.
Most digraphs are selected to give you a hint about the character they will produce.
If you look through the list you will understand the logic.

You can exchange the first and second character, if there is no digraph for that combination.
Thus CTRL-K dP also works.
Since there is no digraph for "\texttt{dP}" Vim will also search for a "\texttt{Pd}" digraph.

\emph{Note}: The digraphs depend on the character set that Vim assumes you are using.
On MS-DOS they are different from MS-Windows.
Always use "\texttt{:digraphs}" to find out which digraphs are currently available.

You can define your own digraphs.  Example:

\begin{Verbatim}[samepage=true]
 :digraph a" ä
\end{Verbatim}

This defines that CTRL-K a" inserts an (ä) character.
You can also specify the character with a decimal number.
This defines the same digraph:

\begin{Verbatim}[samepage=true]
 :digraph a" 228
\end{Verbatim}

More information about digraphs here: |\texttt{:h digraphs}|

Another way to insert special characters is with a keymap.
More about that here: |\hyperref[Entering language text]{\texttt{Entering language text}}|
\subsection{Normal mode commands}
Insert mode offers a limited number of commands.
In Normal mode you have many more.
When you want to use one, you usually leave Insert mode with <Esc>, execute the Normal mode command, and re-enter Insert mode with "\texttt{i}" or "\texttt{a}".

There is a quicker way.
With CTRL-O \texttt{{command}} you can execute any Normal mode command from Insert mode.
For example, to delete from the cursor to the end of the line:

\begin{Verbatim}[samepage=true]
 CTRL-O D
\end{Verbatim}

You can execute only one Normal mode command this way.
But you can specify a register or a count.
A more complicated example:

\begin{Verbatim}[samepage=true]
 CTRL-O "g3dw
\end{Verbatim}

This deletes up to the third word into register \texttt{g}.
\clearpage
