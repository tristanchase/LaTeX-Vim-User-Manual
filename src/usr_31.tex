\section{31. Exploiting the GUI}
Vim works well in a terminal, but the GUI has a few extra items.  A file
browser can be used for commands that use a file.  A dialog to make a choice
between alternatives.  Use keyboard shortcuts to access menu items quickly.
%\localtableofcontentswithrelativedepth{+1} % I couldn't get this to work TMC
\etocsetnexttocdepth{2} % This produces the same result TMC
\etocsettocstyle{\subsection*{Contents}}{}
\localtableofcontents
\subsection{The file browser}
When using the File/Open... menu you get a file browser.
This makes it easier to find the file you want to edit.
But what if you want to split a window to edit another file?  There is no menu entry for this.
You could first use Window/Split and then File/Open..., but that's more work.

Since you are typing most commands in Vim, opening the file browser with a typed command is possible as well.
To make the split command use the file browser, prepend ``\texttt{browse}":

\begin{Verbatim}[samepage=true]
 :browse split
\end{Verbatim}

Select a file and then the ``\texttt{:split}" command will be executed with it.
If you cancel the file dialog nothing happens, the window isn't split.

You can also specify a file name argument.
This is used to tell the file browser where to start.
Example:

\begin{Verbatim}[samepage=true]
 :browse split /etc
\end{Verbatim}

The file browser will pop up, starting in the directory ``\texttt{/etc}".

The ``\texttt{:browse}" command can be prepended to just about any command that opens a file.

If no directory is specified, Vim will decide where to start the file browser.
By default it uses the same directory as the last time.
Thus when you used ``\texttt{:browse split}" and selected a file in ``\texttt{/usr/local/share}", the next time you use a ``\texttt{:browse}" it will start in ``\texttt{/usr/local/share}" again.

This can be changed with the \texttt{'browsedir'} option.  It can have one of three values:

\begin{center} \begin{tabular}{l l}
				\texttt{last} & Use the last directory browsed (default) \\
				\texttt{buffer} & Use the same directory as the current buffer \\
				\texttt{current} & use the current directory \\
\end{tabular} \end{center}

For example, when you are in the directory ``\texttt{/usr}", editing the file ``\texttt{/usr/local/share/readme}", then the command:

\begin{Verbatim}[samepage=true]
 :set browsedir=buffer
 :browse edit
\end{Verbatim}

Will start the browser in ``\texttt{/usr/local/share}".
Alternatively:

\begin{Verbatim}[samepage=true]
 :set browsedir=current
 :browse edit
\end{Verbatim}

Will start the browser in ``\texttt{/usr}".

\emph{Note}: To avoid using the mouse, most file browsers offer using key presses to navigate.
Since this is different for every system, it is not explained here.
Vim uses a standard browser when possible, your system documentation should contain an explanation on the keyboard shortcuts somewhere.

When you are not using the GUI version, you could use the file explorer window to select files like in a file browser.
However, this doesn't work for the ``\texttt{:browse}" command.
See |\texttt{:h netrw-browse}|.
\subsection{Confirmation}
Vim protects you from accidentally overwriting a file and other ways to lose changes.
If you do something that might be a bad thing to do, Vim produces an error message and suggests appending ! if you really want to do it.

To avoid retyping the command with the !, you can make Vim give you a dialog.
You can then press ``OK" or ``Cancel" to tell Vim what you want.

For example, you are editing a file and made changes to it.
You start editing another file with:

\begin{Verbatim}[samepage=true]
 :confirm edit foo.txt
\end{Verbatim}

Vim will pop up a dialog that looks something like this:

\begin{Verbatim}[samepage=true]
    +-----------------------------------+
    |                                   |
    |   ?  Save changes to "bar.txt"?   |
    |                                   |
    |   YES   NO        CANCEL          |
    +-----------------------------------+
\end{Verbatim}

Now make your choice.
If you do want to save the changes, select ``YES".
If you want to lose the changes for ever: ``NO".
If you forgot what you were doing and want to check what really changed use ``CANCEL".
You will be back in the same file, with the changes still there.

Just like ``\texttt{:browse}", the ``\texttt{:confirm}" command can be prepended to most commands that edit another file.
They can also be combined:

\begin{Verbatim}[samepage=true]
 :confirm browse edit
\end{Verbatim}

This will produce a dialog when the current buffer was changed.
Then it will pop up a file browser to select the file to edit.

\emph{Note}: In the dialog you can use the keyboard to select the choice.
Typically the <Tab> key and the cursor keys change the choice.
Pressing <Enter> selects the choice.
This depends on the system though.

When you are not using the GUI, the ``\texttt{:confirm}" command works as well.
Instead of popping up a dialog, Vim will print the message at the bottom of the Vim window and ask you to press a key to make a choice.

\begin{Verbatim}[samepage=true]
 :confirm edit main.c
    Save changes to "Untitled"?
    [Y]es, (N)o, (C)ancel:
\end{Verbatim}

You can now press the single key for the choice.
You don't have to press <Enter>, unlike other typing on the command line.
\subsection{Menu shortcuts}
The keyboard is used for all Vim commands.
The menus provide a simple way to select commands, without knowing what they are called.
But you have to move your hand from the keyboard and grab the mouse.

Menus can often be selected with keys as well.
This depends on your system, but most often it works this way.
Use the <Alt> key in combination with the underlined letter of a menu.
For example, <A-w> (<Alt> and w) pops up the Window menu.

In the Window menu, the ``\texttt{split}" item has the p underlined.
To select it, let go of the <Alt> key and press p.

After the first selection of a menu with the <Alt> key, you can use the cursor keys to move through the menus.
<Right> selects a submenu and <left> closes it.
<Esc> also closes a menu.
<Enter> selects a menu item.

There is a conflict between using the <Alt> key to select menu items, and using <Alt> key combinations for mappings.
The \texttt{'winaltkeys'} option tells Vim what it should do with the <Alt> key.

The default value ``\texttt{menu}" is the smart choice: If the key combination is a menu shortcut it can't be mapped.
All other keys are available for mapping.

The value ``\texttt{no}" doesn't use any <Alt> keys for the menus.
Thus you must use the mouse for the menus, and all <Alt> keys can be mapped.

The value ``\texttt{yes}" means that Vim will use any <Alt> keys for the menus.
Some <Alt> key combinations may also do other things than selecting a menu.
\subsection{Vim window position and size}
To see the current Vim window position on the screen use:

\begin{Verbatim}[samepage=true]
 :winpos
\end{Verbatim}

This will only work in the GUI.
The output may look like this:

\begin{Verbatim}[samepage=true]
    Window position: X 272, Y 103
\end{Verbatim}

The position is given in screen pixels.
Now you can use the numbers to move Vim somewhere else.
For example, to move it to the left a hundred pixels:

\begin{Verbatim}[samepage=true]
 :winpos 172 103
\end{Verbatim}

\emph{Note}: There may be a small offset between the reported position and where the window moves.
This is because of the border around the window.
This is added by the window manager.

You can use this command in your startup script to position the window at a specific position.

The size of the Vim window is computed in characters.
Thus this depends on the size of the font being used.
You can see the current size with this command:

\begin{Verbatim}[samepage=true]
 :set lines columns
\end{Verbatim}

To change the size set the \texttt{'lines'} and/or \texttt{'columns'} options to a new value:

\begin{Verbatim}[samepage=true]
 :set lines=50
 :set columns=80
\end{Verbatim}

Obtaining the size works in a terminal just like in the GUI.
Setting the size is not possible in most terminals.

You can start the X-Windows version of gvim with an argument to specify the size and position of the window:

\begin{Verbatim}[samepage=true]
 gvim -geometry {width}x{height}+{x_offset}+{y_offset}
\end{Verbatim}

\texttt{{width}} and \texttt{{height}} are in characters, \texttt{{x\_offset}} and \texttt{{y\_offset}} are in pixels.
Example:

\begin{Verbatim}[samepage=true]
 gvim -geometry 80x25+100+300
\end{Verbatim}
\subsection{Various}
You can use gvim to edit an e-mail message.
In your e-mail program you must select gvim to be the editor for messages.
When you try that, you will see that it doesn't work: The mail program thinks that editing is finished, while gvim is still running!

What happens is that gvim disconnects from the shell it was started in.
That is fine when you start gvim in a terminal, so that you can do other work in that terminal.
But when you really want to wait for gvim to finish, you must prevent it from disconnecting.
The ``\texttt{-f}" argument does this:

\begin{Verbatim}[samepage=true]
 gvim -f file.txt
\end{Verbatim}

The ``\texttt{-f}" stands for foreground.
Now Vim will block the shell it was started in until you finish editing and exit.

\subsubsection{Delayed start of the gui}
On Unix it's possible to first start Vim in a terminal.
That's useful if you do various tasks in the same shell.
If you are editing a file and decide you want to use the GUI after all, you can start it with:

\begin{Verbatim}[samepage=true]
 :gui
\end{Verbatim}

Vim will open the GUI window and no longer use the terminal.
You can continue using the terminal for something else.
The ``\texttt{-f}" argument is used here to run the GUI in the foreground.
You can also use ``\texttt{:gui -f}".

\subsubsection{The gvim startup file}
When gvim starts, it reads the gvimrc file.
That's similar to the vimrc file used when starting Vim.
The gvimrc file can be used for settings and commands that are only to be used when the GUI is going to be started.
For example, you can set the \texttt{'lines'} option to set a different window size:

\begin{Verbatim}[samepage=true]
 :set lines=55
\end{Verbatim}

You don't want to do this in a terminal, since its size is fixed (except for an xterm that supports resizing).

The gvimrc file is searched for in the same locations as the vimrc file.
Normally its name is ``\texttt{~/.gvimrc}" for Unix and ``\texttt{\$VIM/\_gvimrc}" for MS-Windows.
The \texttt{\$MYGVIMRC} environment variable is set to it, thus you can use this command to edit the file, if you have one:

\begin{Verbatim}[samepage=true]
 :edit $MYGVIMRC
\end{Verbatim}

If for some reason you don't want to use the normal gvimrc file, you can specify another one with the ``\texttt{-U}" argument:

\begin{Verbatim}[samepage=true]
 gvim -U thisrc ...
\end{Verbatim}

That allows starting gvim for different kinds of editing.
You could set another font size, for example.

To completely skip reading a gvimrc file:

\begin{Verbatim}[samepage=true]
 gvim -U NONE ...
\end{Verbatim}
\clearpage
