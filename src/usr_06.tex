\section{06. Using syntax highlighting}
\label{Using syntax highlighting}
Black and white text is boring.  With colors your file comes to life.  This
not only looks nice, it also speeds up your work.  Change the colors used for
the different sorts of text.  Print your text, with the colors you see on the
screen.
\localtableofcontentswithrelativedepth{+1}
\subsection{Switching it on}

It all starts with one simple command:

 \begin{Verbatim}[samepage=true]
 :syntax enable
 \end{Verbatim}

That should work in most situations to get color in your files.
Vim will automagically detect the type of file and load the right syntax highlighting.
Suddenly comments are blue, keywords brown and strings red.
This makes it easy to overview the file.
After a while you will find that black\&white text slows you down!

If you always want to use syntax highlighting, put the "\verb!:syntax enable!" command in your |\verb!:h vimrc!| file.

If you want syntax highlighting only when the terminal supports colors, you can put this in your |\verb!:h vimrc!| file:

 \begin{Verbatim}[samepage=true]
 if &t_Co > 1
    syntax enable
 endif
 \end{Verbatim}

If you want syntax highlighting only in the GUI version, put the "\verb!:syntax enable!" command in your |\verb!:h gvimrc!| file.

\subsection{No or wrong colors?}
\label{No or wrong colors?}

There can be a number of reasons why you don't see colors:
\begin{description}
		% Find other lists like this. This is thw way to format this.
				\item [Your terminal does not support colors.]
								Vim will use bold, italic and underlined text, but this doesn't look very nice.
								You probably will want to try to get a terminal with colors.
								For Unix, I recommend the xterm from the XFree86 project: |\verb!:h xfree-xterm!|.

				\item [Your terminal does support colors, but Vim doesn't know this.]
								Make sure your \verb!$TERM! setting is correct.  For example, when using an
								xterm that supports colors:

								\begin{Verbatim}[samepage=true]
		 setenv TERM xterm-color
								\end{Verbatim}

								or (depending on your shell):

								\begin{Verbatim}[samepage=true]
		 TERM=xterm-color; export TERM
								\end{Verbatim}

								The terminal name must match the terminal you are using.
								If it still doesn't work, have a look at |\verb!:h xterm-color!|, which shows a few ways to make Vim display colors (not only for an xterm).

				\item [The file type is not recognized.]
								Vim doesn't know all file types, and sometimes it's near to impossible
								to tell what language a file uses.  Try this command:

								\begin{Verbatim}[samepage=true]
		 :set filetype
								\end{Verbatim}

								If the result is "filetype=" then the problem is indeed that Vim doesn't know what type of file this is.
								You can set the type manually:

								\begin{Verbatim}[samepage=true]
		 :set filetype=fortran
												\end{Verbatim}

								To see which types are available, look in the directory \verb!$VIMRUNTIME/syntax!.
								For the GUI you can use the Syntax menu.
								Setting the filetype can also be done with a |\verb!:h modeline!|, so that the file will be highlighted each time you edit it.
								For example, this line can be used in a Makefile (put it near the start or end of the file):

								\begin{Verbatim}[samepage=true]
		 # vim: syntax=make

								\end{Verbatim}

								You might know how to detect the file type yourself.
								Often the file name extension (after the dot) can be used.
								See |\verb!:h new-filetype!| for how to tell Vim to detect that file type.

				\item [There is no highlighting for your file type.]
								You could try using a similar file type by manually setting it as mentioned above.
								If that isn't good enough, you can write your own syntax file, see |\verb!:h mysyntaxfile!|.
\end{description}

Or the colors could be wrong:

\begin{description}
				\item [The colored text is very hard to read.]
								Vim guesses the background color that you are using.  If it is black
								(or another dark color) it will use light colors for text.  If it is
								white (or another light color) it will use dark colors for text.  If
								Vim guessed wrong the text will be hard to read.  To solve this, set
								the \verb!'background'! option.  For a dark background:

								\begin{Verbatim}[samepage=true]
		 :set background=dark
												\end{Verbatim}

								And for a light background:

								\begin{Verbatim}[samepage=true]
		 :set background=light
												\end{Verbatim}

								Make sure you put this \emph{before} the "\verb!:syntax enable!" command, otherwise the colors will already have been set.
								You could do "\verb!:syntax reset!" after setting \verb!'background'! to make Vim set the default colors again.

				\item [The colors are wrong when scrolling bottom to top.]
								Vim doesn't read the whole file to parse the text.
								It starts parsing wherever you are viewing the file.
								That saves a lot of time, but sometimes the colors are wrong.
								A simple fix is hitting CTRL-L.
								Or scroll back a bit and then forward again.
								For a real fix, see |\verb!:h :syn-sync!|.
								Some syntax files have a way to make it look further back, see the help for the specific syntax file.
								For example, |\verb!:h tex.vim!| for the TeX syntax.
\end{description}

\subsection{Different colors}
\label{syn-default-override}

If you don't like the default colors, you can select another color scheme.
In the GUI use the Edit/Color Scheme menu.
You can also type the command:

 \begin{Verbatim}[samepage=true]
 :colorscheme evening
 \end{Verbatim}

"evening" is the name of the color scheme.
There are several others you might want to try out.
Look in the directory \verb!$VIMRUNTIME/colors!.

When you found the color scheme that you like, add the "\verb!:colorscheme!" command to your |\verb!:h vimrc!| file.

You could also write your own color scheme.
This is how you do it:
\begin{enumerate}
				\item Select a color scheme that comes close.
								Copy this file to your own Vim directory.
								For Unix, this should work:

								\begin{Verbatim}[samepage=true]
 !mkdir ~/.vim/colors
 !cp $VIMRUNTIME/colors/morning.vim ~/.vim/colors/mine.vim
 \end{Verbatim}

								This is done from Vim, because it knows the value of \verb!$VIMRUNTIME!.

				\item Edit the color scheme file.
								These entries are useful:

								\begin{tabular}{c l}
												term & attributes in a B\&W terminal\\
												cterm & attributes in a color terminal\\
												ctermfg & foreground color in a color terminal\\
												ctermbg & background color in a color terminal\\
												gui & attributes in the GUI\\
												guifg & foreground color in the GUI\\
												guibg & background color in the GUI\\
								\end{tabular}

								For example, to make comments green:

								\begin{Verbatim}[samepage=true]
 :highlight Comment ctermfg=green guifg=green
								\end{Verbatim}

								Attributes you can use for "\verb!cterm!" and "\verb!gui!" are "\verb!bold!" and "\verb!underline!".
								If you want both, use "\verb!bold,underline!".
								For details see the |\verb!:h :highlight!| command.

				\item Tell Vim to always use your color scheme.
								Put this line in your |\verb!:h vimrc!|:

								\begin{Verbatim}[samepage=true]
 colorscheme mine
 \end{Verbatim}

\end{enumerate}
If you want to see what the most often used color combinations look like, use this command:

 \begin{Verbatim}[samepage=true]
 :runtime syntax/colortest.vim
 \end{Verbatim}

You will see text in various color combinations.
You can check which ones are readable and look nice.

\subsection{With colors or without colors}

Displaying text in color takes a lot of effort.
If you find the displaying too slow, you might want to disable syntax highlighting for a moment:

\begin{Verbatim}[samepage=true]
 :syntax clear
\end{Verbatim}

When editing another file (or the same one) the colors will come back.

\phantomsection
\label{:syn-off}
If you want to stop highlighting completely use:

 \begin{Verbatim}[samepage=true]
 :syntax off
 \end{Verbatim}

This will completely disable syntax highlighting and remove it immediately for all buffers.

\phantomsection
\label{:syn-manual}
If you want syntax highlighting only for specific files, use this:

 \begin{Verbatim}[samepage=true]
 :syntax manual
 \end{Verbatim}

This will enable the syntax highlighting, but not switch it on automatically when starting to edit a buffer.
To switch highlighting on for the current buffer, set the \verb!'syntax'! option:

 \begin{Verbatim}[samepage=true]
 :set syntax=ON
 \end{Verbatim}

\subsection{Printing with colors}
\label{syntax-printing}
In the MS-Windows version you can print the current file with this command:

 \begin{Verbatim}[samepage=true]
 :hardcopy
 \end{Verbatim}

You will get the usual printer dialog, where you can select the printer and a few settings.
If you have a color printer, the paper output should look the same as what you see inside Vim.
But when you use a dark background the colors will be adjusted to look good on white paper.

There are several options that change the way Vim prints:
\begin{itemize}
				\item \verb!'printdevice'!
				\item \verb!'printheader'!
				\item \verb!'printfont'!
				\item \verb!'printoptions'!
\end{itemize}

To print only a range of lines, use Visual mode to select the lines and then type the command:

 \begin{Verbatim}[samepage=true]
 v100j:hardcopy
 \end{Verbatim}

"\verb!v!" starts Visual mode.
"\verb!100j!" moves a hundred lines down, they will be highlighted.
Then "\verb!:hardcopy!" will print those lines.
You can use other commands to move in Visual mode, of course.

This also works on Unix, if you have a PostScript printer.
Otherwise, you will have to do a bit more work.
You need to convert the text to HTML first, and then print it from a web browser.

Convert the current file to HTML with this command:

 \begin{Verbatim}[samepage=true]
 :TOhtml
 \end{Verbatim}

In case that doesn't work:

 \begin{Verbatim}[samepage=true]
 :source $VIMRUNTIME/syntax/2html.vim
 \end{Verbatim}

You will see it crunching away, this can take quite a while for a large file.
Some time later another window shows the HTML code.
Now write this somewhere (doesn't matter where, you throw it away later):

 \begin{Verbatim}[samepage=true]
 :write main.c.html
 \end{Verbatim}

Open this file in your favorite browser and print it from there.
If all goes well, the output should look exactly as it does in Vim.
See |\verb!:h 2html.vim!| for details.
Don't forget to delete the HTML file when you are done with it.

Instead of printing, you could also put the HTML file on a web server, and let others look at the colored text.

\subsection{Further reading}
|\hyperref[Your own syntax highlighted]{\texttt{Your own syntax highlighted}}|\\
|\verb!:h syntax!|      All the details.

\clearpage
