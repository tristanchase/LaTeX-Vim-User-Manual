section{26. Repeating}
An editing task is hardly ever unstructured.
A change often needs to be made several times.
In this chapter a number of useful ways to repeat a change will be explained.
%\localtableofcontentswithrelativedepth{+1} % I couldn't get this to work TMC
\etocsetnexttocdepth{2} % This produces the same result TMC
\etocsettocstyle{\subsection*{Contents}}{}
\localtableofcontents
\subsection{Repeating with Visual mode}
Visual mode is very handy for making a change in any sequence of lines.
You can see the highlighted text, thus you can check if the correct lines are changed.
But making the selection takes some typing.
The "\texttt{gv}" command selects the same area again.
This allows you to do another operation on the same text.

Suppose you have some lines where you want to change "2001" to "2002" and "2000" to "2001":

\begin{Verbatim}[samepage=true]
    The financial results for 2001 are better 
    than for 2000.  The income increased by 50%, 
    even though 2001 had more rain than 2000. 
                2000        2001 
    income      45,403      66,234 
\end{Verbatim}

First change "2001" to "2002".  Select the lines in Visual mode, and use:

\begin{Verbatim}[samepage=true]
 :s/2001/2002/g
\end{Verbatim}

Now use "\texttt{gv}" to reselect the same text.
It doesn't matter where the cursor is.
Then use "\texttt{:s/2000/2001/g}" to make the second change.

Obviously, you can repeat these changes several times.
\subsection{Add and subtract}
When repeating the change of one number into another, you often have a fixed offset.
In the example above, one was added to each year.
Instead of typing a substitute command for each year that appears, the CTRL-A command can be used.

Using the same text as above, search for a year:

\begin{Verbatim}[samepage=true]
 /19[0-9][0-9]\|20[0-9][0-9]
\end{Verbatim}

Now press CTRL-A.
The year will be increased by one:

\begin{Verbatim}[samepage=true]
    The financial results for 2002 are better 
    than for 2000.  The income increased by 50%, 
    even though 2001 had more rain than 2000. 
                2000        2001 
    income      45,403      66,234 
\end{Verbatim}

Use "\texttt{n}" to find the next year, and press "\texttt{.}" to repeat the CTRL-A ("\texttt{.}" is a bit quicker to type).
Repeat "\texttt{n}" and "\texttt{.}" for all years that appear.

Hint: set the \texttt{'hlsearch'} option to see the matches you are going to change, then you can look ahead and do it faster.

Adding more than one can be done by prepending the number to CTRL-A.
Suppose you have this list:

\begin{Verbatim}[samepage=true]
    1.  item four 
    2.  item five 
    3.  item six 
\end{Verbatim}

Move the cursor to "1." and type:

\begin{Verbatim}[samepage=true]
 3 CTRL-A
\end{Verbatim}

The "1." will change to "4.".
Again, you can use "\texttt{.}" to repeat this on the other numbers.

Another example:

\begin{Verbatim}[samepage=true]
    006 foo bar 
    007 foo bar 
\end{Verbatim}

Using CTRL-A on these numbers results in:

\begin{Verbatim}[samepage=true]
    007 foo bar 
    010 foo bar 
\end{Verbatim}

7 plus one is 10?  What happened here is that Vim recognized "007" as an octal number, because there is a leading zero.
This notation is often used in C programs.
If you do not want a number with leading zeros to be handled as octal, use this:

\begin{Verbatim}[samepage=true]
 :set nrformats-=octal
\end{Verbatim}

The CTRL-X command does subtraction in a similar way.
\subsection{Making a change in many files}
Suppose you have a variable called "\texttt{x\_cnt}" and you want to change it to "\texttt{x\_counter}".
This variable is used in several of your C files.
You need to change it in all files.
This is how you do it.

Put all the relevant files in the argument list:

\begin{Verbatim}[samepage=true]
 :args *.c
\end{Verbatim}
 
This finds all C files and edits the first one.
Now you can perform a substitution command on all these files:

\begin{Verbatim}[samepage=true]
 :argdo %s/\<x_cnt\>/x_counter/ge | update
\end{Verbatim}

The "\texttt{:argdo}" command takes an argument that is another command.
That command will be executed on all files in the argument list.

The "\texttt{\%s}" substitute command that follows works on all lines.
It finds the word "\texttt{x\_cnt}" with "\texttt{\textbackslash<x\_cnt\textbackslash>}".
The "\texttt{\textbackslash<}" and "\texttt{\textbackslash>}" are used to match the whole word only, and not "\texttt{px\_cnt}" or "\texttt{x\_cnt2}".

The flags for the substitute command include "\texttt{g}" to replace all occurrences of "\texttt{x\_cnt}" in the same line.
The "\texttt{e}" flag is used to avoid an error message when "\texttt{x\_cnt}" does not appear in the file.
Otherwise "\texttt{:argdo}" would abort on the first file where "\texttt{x\_cnt}" was not found.

The "\texttt{|}" separates two commands.
The following "\texttt{update}" command writes the file only if it was changed.
If no "\texttt{x\_cnt}" was changed to "\texttt{x\_counter}" nothing happens.

There is also the "\texttt{:windo}" command, which executes its argument in all windows.
And "\texttt{:bufdo}" executes its argument on all buffers.
Be careful with this, because you might have more files in the buffer list than you think.
Check this with the "\texttt{:buffers}" command (or "\texttt{:ls}").
\subsection{Using Vim from a shell script}
Suppose you have a lot of files in which you need to change the string "-person-" to "Jones" and then print it.
How do you do that?  One way is to do a lot of typing.
The other is to write a shell script to do the work.

The Vim editor does a superb job as a screen-oriented editor when using Normal mode commands.
For batch processing, however, Normal mode commands do not result in clear, commented command files; so here you will use Ex mode instead.
This mode gives you a nice command-line interface that makes it easy to put into a batch file.
("Ex command" is just another name for a command-line (:) command.)

The Ex mode commands you need are as follows:

\begin{Verbatim}[samepage=true]
 %s/-person-/Jones/g
 write tempfile
 quit
\end{Verbatim}

You put these commands in the file "\texttt{change.vim}".
Now to run the editor in batch mode, use this shell script:

\begin{Verbatim}[samepage=true]
 for file in *.txt; do
   vim -e -s $file < change.vim
   lpr -r tempfile
 done
\end{Verbatim}

The for-done loop is a shell construct to repeat the two lines in between, while the \texttt{\$file} variable is set to a different file name each time.

The second line runs the Vim editor in Ex mode (-e argument) on the file \texttt{\$file} and reads commands from the file "\texttt{change.vim}".
The -s argument tells Vim to operate in silent mode.
In other words, do not keep outputting the :prompt, or any other prompt for that matter.

The "\texttt{lpr -r tempfile}" command prints the resulting "\texttt{tempfile}" and deletes it (that's what the -r argument does).
\subsubsection{Reading from stdin}
Vim can read text on standard input.
Since the normal way is to read commands there, you must tell Vim to read text instead.
This is done by passing the "\texttt{-}" argument in place of a file.
Example:

\begin{Verbatim}[samepage=true]
 ls | vim -
\end{Verbatim}

This allows you to edit the output of the "\texttt{ls}" command, without first saving the text in a file.

If you use the standard input to read text from, you can use the "\texttt{-S}" argument to read a script:

\begin{Verbatim}[samepage=true]
 producer | vim -S change.vim -
\end{Verbatim}

\subsubsection{Normal mode scripts}
If you really want to use Normal mode commands in a script, you can use it like this:

\begin{Verbatim}[samepage=true]
 vim -s script file.txt ...
\end{Verbatim}
 
\emph{Note}: "\texttt{-s}" has a different meaning when it is used without "\texttt{-e}".
Here it means to source the "\texttt{script}" as Normal mode commands.
When used with "\texttt{-e}" it means to be silent, and doesn't use the next argument as a file name.

The commands in "\texttt{script}" are executed like you typed them.
Don't forget that a line break is interpreted as pressing <Enter>.
In Normal mode that moves the cursor to the next line.

To create the script you can edit the script file and type the commands.
You need to imagine what the result would be, which can be a bit difficult.
Another way is to record the commands while you perform them manually.
This is how you do that:

\begin{Verbatim}[samepage=true]
 vim -w script file.txt ...
\end{Verbatim}

All typed keys will be written to "\texttt{script}".
If you make a small mistake you can just continue and remember to edit the script later.

The "\texttt{-w}" argument appends to an existing script.
That is good when you want to record the script bit by bit.
If you want to start from scratch and start all over, use the "\texttt{-W}" argument.
It overwrites any existing file.
\clearpage
