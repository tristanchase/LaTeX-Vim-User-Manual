\section{02. The first steps in Vim}
This chapter provides just enough information to edit a file with Vim.  Not
well or fast, but you can edit.  Take some time to practice with these
commands, they form the base for what follows.
\localtableofcontentswithrelativedepth{+1}

\subsection{Running Vim for the First Time}
To start Vim, enter this command:

\begin{Verbatim}[samepage=true]
 gvim file.txt
\end{Verbatim}

In UNIX you can type this at any command prompt.
If you are running Microsoft Windows, open an MS-DOS prompt window and enter the command.
In either case, Vim starts editing a file called file.txt.
Because this is a new file, you get a blank window.
This is what your screen will look like:

\begin{Verbatim}[samepage=true]
		+---------------------------------------+
		|#                                      |
		|~                                      |
		|~                                      |
		|~                                      |
		|~                                      |
		|"file.txt" [New file]                  |
		+---------------------------------------+
				('#" is the cursor position.)
\end{Verbatim}

The tilde (\textasciitilde) lines indicate lines not in the file.
In other words, when Vim runs out of file to display, it displays tilde lines.
At the bottom of the screen, a message line indicates the file is named file.txt and shows that you are creating a new file.
The message information is temporary and other information overwrites it.

\subsubsection{The vim command}

The gvim command causes the editor to create a new window for editing.
If you use this command:

 \begin{Verbatim}[samepage=true]
 vim file.txt
 \end{Verbatim}

the editing occurs inside your command window.
In other words, if you are running inside an xterm, the editor uses your xterm window.
If you are using an MS-DOS command prompt window under Microsoft Windows, the editing occurs inside this window.
The text in the window will look the same for both versions, but with gvim you have extra features, like a menu bar.
More about that later.

\subsection{Inserting text}
The Vim editor is a modal editor.
That means that the editor behaves differently, depending on which mode you are in.
The two basic modes are called Normal mode and Insert mode.
In Normal mode the characters you type are commands.
In Insert mode the characters are inserted as text.
Since you have just started Vim it will be in Normal mode.
To start Insert mode you type the "i" command (i for Insert).
Then you can enter the text.
It will be inserted into the file.
Do not worry if you make mistakes; you can correct them later.
To enter the following programmer's limerick, this is what you type:

 \begin{Verbatim}[samepage=true]
 iA very intelligent turtle
 Found programming UNIX a hurdle
 \end{Verbatim}

After typing "turtle" you press the <Enter> key to start a new line.
Finally you press the <Esc> key to stop Insert mode and go back to Normal mode.
You now have two lines of text in your Vim window: 

		\begin{Verbatim}[samepage=true]
		+---------------------------------------+
		|A very intelligent turtle              |
		|Found programming UNIX a hurdle        |
		|~                                      |
		|~                                      |
		|                                       |
		+---------------------------------------+
		\end{Verbatim}

\subsubsection{What is the mode?}

To be able to see what mode you are in, type this command:

 \begin{Verbatim}[samepage=true]
 :set showmode
 \end{Verbatim}

You will notice that when typing the colon Vim moves the cursor to the last line of the window.
That's where you type colon commands (commands that start with a colon).
Finish this command by pressing the <Enter> key (all commands that start with a colon are finished this way).
Now, if you type the "i" command Vim will display --INSERT-- at the bottom of the window.
This indicates you are in Insert mode.

		\begin{Verbatim}[samepage=true]
		+---------------------------------------+
		|A very intelligent turtle              |
		|Found programming UNIX a hurdle        |
		|~                                      |
		|~                                      |
		|-- INSERT --                           |
		+---------------------------------------+
		\end{Verbatim}

If you press <Esc> to go back to Normal mode the last line will be made blank.

\subsubsection{Getting out of trouble}

One of the problems for Vim novices is mode confusion, which is caused by forgetting which mode you are in or by accidentally typing a command that switches modes.
To get back to Normal mode, no matter what mode you are in, press the <Esc> key.
Sometimes you have to press it twice.
If Vim beeps back at you, you already are in Normal mode.

\subsection{Moving around}

After you return to Normal mode, you can move around by using these keys:
\phantomsection
\label{hjlk}
\begin{center}
				\begin{tabular}{c|c} 
								h & left\\
								j & down\\
								k & up\\
								l & right
				\end{tabular}
\end{center}

At first, it may appear that these commands were chosen at random.
After all, who ever heard of using l for right?  But actually, there is a very good reason for these choices: Moving the cursor is the most common thing you do in an editor, and these keys are on the home row of your right hand.
In other words, these commands are placed where you can type them the fastest (especially when you type with ten fingers).

Note:
You can also move the cursor by using the arrow keys.
If you do, however, you greatly slow down your editing because to press the arrow keys, you must move your hand from the text keys to the arrow keys.
Considering that you might be doing it hundreds of times an hour, this can take a significant amount of time.
Also, there are keyboards which do not have arrow keys, or which locate them in unusual places; therefore, knowing the use of the hjkl keys helps in those situations.

One way to remember these commands is that h is on the left, l is on the right and j points down.
In a picture: 

\begin{Verbatim}[samepage=true]
				  k
				h   l
				  j
\end{Verbatim}

The best way to learn these commands is by using them.
Use the "i" command to insert some more lines of text.
Then use the hjkl keys to move around and insert a word somewhere.
Don't forget to press <Esc> to go back to Normal mode.
The \hyperref[vimtutor]{|\texttt{vimtutor}|} is also a nice way to learn by doing.

For Japanese users, Hiroshi Iwatani suggested using this:

\begin{Verbatim}[samepage=true]
                Komsomolsk
                    ^
                    |
       Huan Ho  <--- --->  Los Angeles
    (Yellow river)  |
                    v
                  Java (the island, not the programming language)
\end{Verbatim}

\subsection{Deleting characters}

To delete a character, move the cursor over it and type "x".
(This is a throwback to the old days of the typewriter, when you deleted things by typing xxxx over them.)
Move the cursor to the beginning of the first line, for example, and type xxxxxxx (seven x's) to delete "A very ".
The result should look like this: 

		\begin{Verbatim}[samepage=true]
		+---------------------------------------+
		|intelligent turtle                     |
		|Found programming UNIX a hurdle        |
		|~                                      |
		|~                                      |
		|                                       |
		+---------------------------------------+
		\end{Verbatim}

Now you can insert new text, for example by typing:

 \begin{Verbatim}[samepage=true]
 iA young <Esc>
 \end{Verbatim}

This begins an insert (the i), inserts the words "A young", and then exits insert mode (the final <Esc>).
 The result: 
		\begin{Verbatim}[samepage=true]
		+---------------------------------------+
		|A young intelligent turtle             |
		|Found programming UNIX a hurdle        |
		|~                                      |
		|~                                      |
		|                                       |
		+---------------------------------------+
		\end{Verbatim}

\subsubsection{Deleting a line}

To delete a whole line use the "dd" command.
The following line will then move up to fill the gap: 

		\begin{Verbatim}[samepage=true]
		+---------------------------------------+
		|Found programming UNIX a hurdle        |
		|~                                      |
		|~                                      |
		|~                                      |
		|                                       |
		+---------------------------------------+
		\end{Verbatim}

\subsubsection{Deleting a line break}

In Vim you can join two lines together, which means that the line break between them is deleted.
The "J" command does this.
Take these two lines: 

		\begin{Verbatim}[samepage=true]
		A young intelligent 
		turtle 
		\end{Verbatim}

Move the cursor to the first line and press "J":

		\begin{Verbatim}[samepage=true]
		A young intelligent turtle 
		\end{Verbatim}

\subsection{Undo and Redo}
\label{Undo and Redo}

Suppose you delete too much.
Well, you can type it in again, but an easier way exists.
The "u" command undoes the last edit.
Take a look at this in action: After using "dd" to delete the first line, "u" brings it back.
Another one: Move the cursor to the A in the first line: 

		\begin{Verbatim}[samepage=true]
		A young intelligent turtle 
						\end{Verbatim}

Now type xxxxxxx to delete "A young".
The result is as follows: 

	\begin{Verbatim}[samepage=true]
	intelligent turtle 
	\end{Verbatim}

Type "u" to undo the last delete.
That delete removed the g, so the undo restores the character.

	\begin{Verbatim}[samepage=true]
	g intelligent turtle 
	\end{Verbatim}

The next u command restores the next-to-last character deleted:

	\begin{Verbatim}[samepage=true]
	ng intelligent turtle 
	\end{Verbatim}

The next u command gives you the u, and so on:

	\begin{Verbatim}[samepage=true]
	ung intelligent turtle 
	oung intelligent turtle 
	young intelligent turtle 
	young intelligent turtle 
	A young intelligent turtle 
	\end{Verbatim}

Note:
If you type "u" twice, and the result is that you get the same text back, you have Vim configured to work Vi compatible.
Look here to fix this: \hyperref[not-compatible]{|\texttt{not-compatible}|}.
This text assumes you work "The Vim Way".
You might prefer to use the good old Vi way, but you will have to watch out for small differences in the text then.

\subsubsection{Redo}

If you undo too many times, you can press CTRL-R (redo) to reverse the preceding command.
In other words, it undoes the undo.
To see this in action, press CTRL-R twice.
The character A and the space after it disappear: 

		\begin{Verbatim}[samepage=true]
    young intelligent turtle 
		\end{Verbatim}

There's a special version of the undo command, the "U" (undo line) command.
The undo line command undoes all the changes made on the last line that was edited.
Typing this command twice cancels the preceding "U".

		\begin{Verbatim}[samepage=true]
    A very intelligent turtle 
      xxxx                      Delete very

    A intelligent turtle 
                  xxxxxx        Delete turtle

    A intelligent 
                                Restore line with "U"
    A very intelligent turtle 
                                Undo "U" with "u"
    A intelligent 
		\end{Verbatim}

The "U" command is a change by itself, which the "u" command undoes and CTRL-R redoes.
This might be a bit confusing.
Don't worry, with "u" and CTRL-R you can go to any of the situations you had.
More about that in section |\hyperref[Numbering changes]{\texttt{Numbering changes}}|.

\subsection{Other editing commands}

Vim has a large number of commands to change the text.
See |\verb!:h Q_in!| and below.
Here are a few often used ones.

\subsubsection{Appending}

The "i" command inserts a character before the character under the cursor.
That works fine; but what happens if you want to add stuff to the end of the line?  For that you need to insert text after the cursor.
This is done with the "a" (append) command.
For example, to change the line 

		\begin{Verbatim}[samepage=true]
    and that's not saying much for the turtle. 
		\end{Verbatim}
to
		\begin{Verbatim}[samepage=true]
    and that's not saying much for the turtle!!! 
		\end{Verbatim}

move the cursor over to the dot at the end of the line. Then type "x" to delete the period.
The cursor is now positioned at the end of the line on the e in turtle.
Now type 

	\begin{Verbatim}[samepage=true]
	a!!!<Esc>
	\end{Verbatim}

to append three exclamation points after the e in turtle:

	\begin{Verbatim}[samepage=true]
	and that's not saying much for the turtle!!! 
	\end{Verbatim}

\subsubsection{Opening up a new line}

The "o" command creates a new, empty line below the cursor and puts Vim in Insert mode.
Then you can type the text for the new line.
Suppose the cursor is somewhere in the first of these two lines: 

		\begin{Verbatim}[samepage=true]
    A very intelligent turtle 
    Found programming UNIX a hurdle 
		\end{Verbatim}

If you now use the "o" command and type new text:

		\begin{Verbatim}[samepage=true]
		oThat liked using Vim<Esc>
		\end{Verbatim}

The result is:

		\begin{Verbatim}[samepage=true]
    A very intelligent turtle 
    That liked using Vim 
    Found programming UNIX a hurdle 
		\end{Verbatim}

The "O" command (uppercase) opens a line above the cursor.

\subsubsection{Using a count}

Suppose you want to move up nine lines.
You can type "kkkkkkkkk" or you can enter the command "9k".
In fact, you can precede many commands with a number.
Earlier in this chapter, for instance, you added three exclamation points to the end of a line by typing "a!!!<Esc>".
Another way to do this is to use the command "3a!<Esc>".
The count of 3 tells the command that follows to triple its effect.
Similarly, to delete three characters, use the command "3x".
The count always comes before the command it applies to.

\subsection{Getting out}

To exit, use the "\verb!ZZ!" command.  This command writes the file and exits.

Note:\newline
Unlike many other editors, Vim does not automatically make a backup file.
If you type "\verb!ZZ!", your changes are committed and there's no turning back.
You can configure the Vim editor to produce backup files, see |\hyperref[Backup files]{\texttt{Backup files}}|.

\subsubsection{Discarding changes}

Sometimes you will make a sequence of changes and suddenly realize you were better off before you started.
Not to worry; Vim has a quit-and-throw-things-away command.
It is: 

	\begin{Verbatim}[samepage=true]
	:q!
	\end{Verbatim}

Don't forget to press <Enter> to finish the command.

For those of you interested in the details, the three parts of this command are the colon (:), which enters Command-line mode; the q command, which tells the editor to quit; and the override command modifier (!).
The override command modifier is needed because Vim is reluctant to throw away changes.
If you were to just type "\verb!:q!", Vim would display an error message and refuse to exit: 

		\begin{Verbatim}[samepage=true]
    E37: No write since last change (use ! to override) 
		\end{Verbatim}

By specifying the override, you are in effect telling Vim, "I know that what I'm doing looks stupid, but I'm a big boy and really want to do this."

If you want to continue editing with Vim: The "\verb%:e!%" command reloads the original version of the file.

\subsection{Finding help}

Everything you always wanted to know can be found in the Vim help files.
Don't be afraid to ask!
To get generic help use this command:

 \begin{Verbatim}[samepage=true]
 :help
 \end{Verbatim}

You could also use the first function key <F1>.
If your keyboard has a <Help> key it might work as well.
If you don't supply a subject, "\verb!:help!" displays the general help window.
The creators of Vim did something very clever (or very lazy) with the help system: They made the help window a normal editing window.
You can use all the normal Vim commands to move through the help information.
Therefore h, j, k, and l move left, down, up and right.
To get out of the help window, use the same command you use to get out of the editor: "\verb!ZZ!".
This will only close the help window, not exit Vim.

As you read the help text, you will notice some text enclosed in vertical bars (for example, |\verb!:h help!|).
This indicates a hyperlink.
If you position the cursor anywhere between the bars and press CTRL-] (jump to tag), the help system takes you to the indicated subject.
(For reasons not discussed here, the Vim terminology for a hyperlink is tag.
So CTRL-] jumps to the location of the tag given by the word under the cursor.) After a few jumps, you might want to go back.
CTRL-T (pop tag) takes you back to the preceding position.
CTRL-O (jump to older position) also works nicely here.
At the top of the help screen, there is the notation *help.txt*.
This name between "*" characters is used by the help system to define a tag (hyperlink destination).
See |\hyperref[Using tags]{\texttt{Using tags}}| for details about using tags.

To get help on a given subject, use the following command:

 \begin{Verbatim}[samepage=true]
 :help {subject}
 \end{Verbatim}

To get help on the "x" command, for example, enter the following:

 \begin{Verbatim}[samepage=true]
 :help x
 \end{Verbatim}

To find out how to delete text, use this command:

 \begin{Verbatim}[samepage=true]
 :help deleting
 \end{Verbatim}

To get a complete index of all Vim commands, use the following command:

 \begin{Verbatim}[samepage=true]
 :help index
 \end{Verbatim}

When you need to get help for a control character command (for example, CTRL-A), you need to spell it with the prefix "CTRL-".

 \begin{Verbatim}[samepage=true]
 :help CTRL-A
 \end{Verbatim}

The Vim editor has many different modes.
By default, the help system displays the normal-mode commands.
For example, the following command displays help for the normal-mode CTRL-H command: 

 \begin{Verbatim}[samepage=true]
 :help CTRL-H
 \end{Verbatim}

To identify other modes, use a mode prefix.
If you want the help for the insert-mode version of a command, use "\verb!i_!".
For CTRL-H this gives you the following command: 

 \begin{Verbatim}[samepage=true]
 :help i_CTRL-H
 \end{Verbatim}

When you start the Vim editor, you can use several command-line arguments.
These all begin with a dash (-).
To find what the -t argument does, for example, use the command: 

 \begin{Verbatim}[samepage=true]
 :help -t
 \end{Verbatim}

The Vim editor has a number of options that enable you to configure and customize the editor.
If you want help for an option, you need to enclose it in single quotation marks.
To find out what the \verb!'number'! option does, for example, use the following command: 

 \begin{Verbatim}[samepage=true]
 :help 'number'
 \end{Verbatim}

The table with all mode prefixes can be found here: |\verb!:h help-context!|.

Special keys are enclosed in angle brackets.
To find help on the up-arrow key in Insert mode, for instance, use this command: 

 \begin{Verbatim}[samepage=true]
 :help i_<Up>
 \end{Verbatim}

If you see an error message that you don't understand, for example:

	\begin{Verbatim}[samepage=true]
  E37: No write since last change (use ! to override) 
	\end{Verbatim}

You can use the error ID at the start to find help about it:

 \begin{Verbatim}[samepage=true]
 :help E37
 \end{Verbatim}

\subsubsection{Summary}
\label{help-summary}
\begin{tabularx}{\textwidth}{|c | X|}
				\hline
				\texttt{:help} & 
				Gives you very general help.
				Scroll down to see a list of all helpfiles, including those added locally (i.e. not distributed with Vim). \\ \hline
				\texttt{:help user-toc.txt} & Table of contents of the User Manual.\\ \hline
				\texttt{:help :subject} & Ex-command "subject", for instance the following:\\ \hline
				\texttt{:help :help} & Help on getting help.\\ \hline
				\texttt{:help abc} & normal-mode command "abc".\\ \hline
				\texttt{:help CTRL-B} & Control key <C-B> in Normal mode.\\ \hline
				\texttt{:help i\_abc}  \texttt{:help i\_CTRL-B} & The same in Insert mode. \\ \hline
				\texttt{:help v\_abc}  \texttt{:help v\_CTRL-B} & The same in Visual mode. \\ \hline
				\texttt{:help c\_abc}  \texttt{:help c\_CTRL-B} & The same in Command-line mode. \\ \hline
				\texttt{:help 'subject'} & Option \verb!'subject'!.\\ \hline
				\texttt{:help subject()} & Function "subject".\\ \hline
				\texttt{:help -subject} & Command-line option "-subject".\\ \hline
				\texttt{:help +subject} & Compile-time feature "+subject".\\ \hline
				\texttt{:help EventName} & Autocommand event "EventName".\\ \hline
				\texttt{:help digraphs.txt} & The top of the helpfile "digraph.txt". Similarly for any other helpfile.\\ \hline
				\texttt{:help pattern<Tab>} & Find a help tag starting with "pattern". Repeat <Tab> for others.\\ \hline
				\texttt{:help pattern<Ctrl-D>} & See all possible help tag matches "pattern" at once.\\ \hline
				\texttt{:helpgrep pattern} & Search the whole text of all help files for pattern "pattern".  Jumps to the first match.
				Jump to other matches with:
				\begin{itemize}
								\item \texttt{:cn} next match
								\item \texttt{:cprev :cN} previous match
								\item \texttt{:cfirst :clast} first or last match
								\item \texttt{:copen :cclose} open/close the quickfix window; press <Enter> to jump to the item under the cursor
				\end{itemize}
				\\ \hline
\end{tabularx}
\clearpage
